\documentclass[10pt,a4paper]{article}
\usepackage[utf8]{inputenc}
\usepackage[T1]{fontenc}
\usepackage{amsmath}
\usepackage{amsfonts}
\usepackage{amssymb}
\usepackage{graphicx}
\usepackage{hyperref}
\usepackage[margin=1cm]{geometry}
\setcounter{secnumdepth}{0}

\begin{document}
	\title{Mushroom Risotto}
	\author{Zak Karimjee}
	\maketitle
	\begin{flushright}
		Serves 4 - cooking time 30-40 minutes
	\end{flushright}
	\section{Ingredients}
	\begin{itemize}
		\item \textbf{Risotto rice - 1 cup or 250 grams} - any sticky sort of rice should work, I've done this with paella rice!
		\item \textbf{Stock - 1 Litre} - any stock will do, I use chicken or vegetable normally
		\item \textbf{Mushrooms - 250 grams} - chestnut are good here as they don't wilt as much
		\item \textbf{Spinach or other leafy green} - optional!
		\item \textbf{Onions - 1 large} - or 2 small!
		\item \textbf{Garlic - 4 cloves}
		\item \textbf{Parsley - two handfuls} - fresh ideal, but dried is fine!
		\item \textbf{Parmesan or other hard cheese - grated} - the more the merrier.
		\item \textbf{Lemon juice - half a lemon or 1 tablespoon from a bottle}
		\item \textbf{White wine vinegar - 2-3 tablespoons} - can also use ~200ml of white wine, replacing equivalent volume of the stock.
		\item \textbf{Butter or olive oil} - quantity of this really depends how indulgent you're feeling! I tend to use a bit of oil to fry the onions, then add butter in when toasting the rice
		\item \textbf{Salt \& Pepper} - to season
	\end{itemize}

	\section{Method}
	\begin{enumerate}
		\item Mix stock, vinegar/wine and rice together, stirring to release starch from rice. After a minute, drain the rice off and keep the liquid aside. Stock should be warm preferably but cold is also fine!
		\item Dice onions \& chop or mince garlic. Onions should be medium size - not too fine. Fry on medium heat in half tablespoon of oil or equivalent butter until they start to soften - 3-5 minutes should be fine.
		\item Add in dry rice and toast at medium heat, stirring frequently. Add in the rest of your butter here if you're using it! Cook for around 5 minutes, until rice starts to brown slightly and mixture is dry.
		\item Stir and then pour on $\frac{3}{4}$ of the stock/vinegar liquid. Increase heat until it's simmering; stir rice once; turn heat to lowest possible and cover, leaving rice for 10 minutes. 
		\item Now is a good time to chop the mushrooms into medium pieces - I cut medium size chestnut mushrooms into 8. If you've got a leafy green then chop this up too - spinach can stay in big pieces, but something that won't wilt as much should be chopped more finely.
		\item After the 10 minutes is up stir gently to redistribute the rice, and cook for approximately 5 more minutes.
		\item Now add in the mushrooms, greens and parsley, and pour on the rest of the stock - if you think it might be getting too wet, keep a little aside for later. Add some salt and pepper now as well. Increase the heat slightly and stir often.
		\item Depending on your mushrooms and greens this step will take 5-10 minutes to get them cooked - they don't want to soften up too much. Make sure you keep stirring to stop the mixture sticking to the bottom of the pan. As it cooks, the liquid should slowly absorb and you should be left with a thick sauce coating the rice - if it looks  a bit dry you can add some of the leftover stock or a splash of hot water.
		\item Once you're happy that the mushrooms and greens have cooked, stir in the cheese and lemon juice. Serve on plates and add a spoon of yoghurt if you want to lighten it up a bit!
	\end{enumerate}

	\section{Notes}
	
	This recipe is based on two sources. To cook the risotto rice nicely, I used \cite{perfris} from Serious Eats, which I discovered from an experimental article figuring out the best way to cook risotto \cite{foodlab}. This article discovered a few things: 
	\begin{enumerate}
		\item If you wash the risotto in the stock, you get rid of the starchy layer which allows it to toast nicely. You can then reintroduce the starch later to get the sticky sauce.
		\item You don't need to pour the stock in bit-by-bit as most recipes suggest - you can just dump most of it in and then keep a bit for the end to fine-tune the texture.
		\item You should use the widest pan you can to evenly cook the rice - but this does mean you need to pay attention to stirring!
	\end{enumerate}
	The 'scientific' approach of the food lab article is what made me listen to it, and it turns out very well!	


	The source for the mushroom part is my friend Angus - he's the one who told me that I can add the mushrooms and greens in fairly late and they will still come out fine. I also watched him throw in unhealthy amounts of parmesan and butter, which I've tried to moderate down - but I've discovered it really helps a lot!
	
	My own additions are the lemon juice and parsley - this helps to lighten the dish, because the mushrooms and cheese can make it feel quite heavy and stodgy.


	All in all I think this works really well - you get the beautiful 'lava' texture of rice from the food lab, the lovely cheesy mushroom taste, and a bit of lightness from the parsley and lemon.
	
	\begin{thebibliography}{5}
		\bibitem{perfris} J. Kenji López-Alt - Serious Eats - Perfect Risotto Recipe \url{https://www.seriouseats.com/recipes/2011/10/how-to-make-perfect-risotto-recipe.html}
		
		\bibitem{foodlab}  J. Kenji López-Alt - Serious Eats - The Food Lab: The Road To Better Risotto \url{https://www.seriouseats.com/2011/10/the-food-lab-the-science-of-risotto.html}
		
	\end{thebibliography}
\end{document}